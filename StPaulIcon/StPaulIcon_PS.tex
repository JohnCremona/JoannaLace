\documentclass[a4paper,12pt]{article}
\usepackage{hyperref}
\usepackage{times}
\usepackage{comment}
\usepackage{titlesec}
\usepackage[pdftex]{graphicx}

\usepackage[a4paper, includefoot]{geometry}
\geometry{lmargin=1.5cm,rmargin=1.5cm,top=1cm,height=28cm}
\renewcommand{\thesubsection}{\Roman{subsection}}

\begin{document}

\title{POSTSCRIPT\\\normalsize to\\THE ICON OF ST. PAUL at the Wignacourt Museum, Rabat,
  Malta} 
\author{Joanna Lace} 
\date{}%May 2011}
\maketitle 

Following the Conservation and Restoration Report (Salve Pater Paule,
May 2009, pp.~102--3), further studies have investigated both the
original context of the panel, and the particular style of the sword
carried by St.~Paul.

Firstly, as the present panel appears to represent the left section of
a larger work, unless the latter was a narrative scene, St.~Paul's
stance, facing to our right, implies the following:

(a) The original work would not have shown him paired with St.~Peter
in illustration of the words ``St.~Peter and St.~Paul'', because
St.~Peter is accordingly shown first, and unless they both face
forwards, they turn towards each other.  The same applies when the two
are shown on either side of Christ, a crucifix, or the Virgin and
Child: each turns to the central motif.

(b) In the absence of evidence of any similar cut down the left side
of our panel, we may also discount both an original horizontal
iconostasis beam, and the panel of a polyptic, where St.~Paul was just
one in a long line of saints on the left flank of a central motif.

(c) His right-facing stance does suggest two contexts: either the
panel once formed the left section of an original ex voto composition
in which St.~Paul and another saint were shown to the left and right
respectively of a central motif (as for example in Salvatore Litard's
ex voto with St.~Genevieve, presented to the Sanctuary of Our Lady,
Mellieha (Catalogue no. 24, Salve Pater Paule); or it is the left
section of a painting of two single saints, St.~Paul and a close
associate.  For the latter we have the example of the painting by the
sixteenth century artist Michele Damaskinos, much-travelled Cretan
artist of the second half of the sixteenth century: this shows
St.~Paul with his devoted friend, the Cypriot apostle St.~Barnabus,
companion on his first journey, who later became the patron saint of
their first destination, the island of Cyprus.

Secondly, St.~Paul holds a ceremonial or 'parade' sword which has the
long slender blade of a warrior saint, a type designated by
R.E. Oakeshott\footnote{R. Ewart Oakeshott, ``The Archeology of
  Weapons'', ed. Dover 1966, Chap XVII; idem, ``Records of the
  Medieval Sword'', ed. Boydell \& Brewer 2009, p.216 ff.} as Type
XVII, found all over Europe, and particularly frequently between 1370
and 1425.  By the end of the fifteenth century the hilts of Type XVII
became very varied and their constituent parts---pommel, grip, and
guard---extremely complex.  The pommel of St.~Paul's version is Type
VU: the particular version of pear-shaped pommel known as `key-shaped'
(i.e. a nineteenth-century watch key).\footnote{R. Ewart Oakeshott,
  ``The Sword in the Age of Chivalry'', ed. Boydell \& Brewer 1997,
  p.107ff, pp.139--140, Appendix: Postscript to ed.~1994.}
Damaskinos, the sixteenth-century artist mentioned above, includes it
in his icon of Saints Servius and Bacchus, now in Corfu. In addition,
both pommel and short grip of St.~Paul's sword have a covering of
gilded stucco, features shared with the ceremonial sword of the
Emperor Sigismund~I, made in 1435\footnote{R. Ewart Oakeshott,
op.cit. note 2 above, Plate 42B.}.  And the guard, strongly
recurved, is an example of the decorative designs, immensely varied
and competitive, that had evolved from the simple so-called 'cross
guard' of earlier Crusader times.  In the case of Sigismund, for
instance, the guard has a dragon or ``Wurm'' design - the emblem of
the Gesellschaft der Lindwurms of which he was a
member\footnote{R. Ewart Oakeshott, op.cit. note 3 above, p.126.}.

\bigskip
\noindent
\copyright\ Joanna Lace, May 2011.
\end{document}
